# Домашнее задание №2: Решение нелинейных уравнений (вариант 16)

## 📌 Постановка задачи

Решить два нелинейных уравнения:

1. Найти корень уравнения:
   \[
   f(x) = x^4 - 18x^2 + 6 = 0
   \]
   методом половинного деления (бисекций) с точностью \( \varepsilon = 0.001 \)

2. Найти отличный от нуля корень уравнения:
   \[
   f(x) = 5^x - 2x - 3 = 0
   \]
   методами:
   - простой итерации,
   - Ньютона,
   
   с точностью \( \varepsilon = 0.0001 \)

---

## ⚙️ Используемые методы и условия их сходимости

### 🔹 Метод бисекций (половинного деления)

Итерационный метод для поиска корня уравнения на интервале, где функция меняет знак.

- **Условие применимости**: функция \( f(x) \) непрерывна на [a, b], и \( f(a) \cdot f(b) < 0 \)
- **Сходимость**: гарантирована при вышеуказанном условии
- **Критерий остановки**: \( \frac{b - a}{2} < \varepsilon \)

---

### 🔹 Метод простой итерации

Преобразует исходное уравнение \( f(x) = 0 \) в эквивалентное \( x = \varphi(x) \), и затем решает \( x_{n+1} = \varphi(x_n) \).

- **Условие сходимости**:
  - \( \varphi(x) \) — непрерывна на [a, b]
  - \( |\varphi'(x)| \leq q < 1 \) (т.е. φ должна быть **сжимающим отображением**)
- **Критерий остановки**:
  - \( |x_{n+1} - x_n| < \varepsilon \)

---

### 🔹 Метод Ньютона

Итерационный метод с квадратичной скоростью сходимости:

\[
x_{n+1} = x_n - \frac{f(x_n)}{f'(x_n)}
\]

- **Условия сходимости**:
  - \( f(x) \in C^2[a,b] \)
  - \( f'(x_0) \ne 0 \)
  - начальное приближение достаточно близко к корню
- **Критерий остановки**:
  - \( |x_{n+1} - x_n| < \varepsilon \)

---

## ✅ Как мы подбирали φ(x)

Для уравнения:
\[
5^x - 2x - 3 = 0
\]

Наивная попытка взять:
\[
x = \varphi(x) = \frac{5^x - 3}{2}
\]
оказалась **неудачной**: при \( x_0 = 2 \) уже на втором шаге возникает переполнение (OverflowError), т.к. \( 5^x \) растёт слишком быстро.

### 📈 Улучшенное преобразование:
Взяв логарифм обеих частей:
\[
5^x = 2x + 3 \Rightarrow x \cdot \ln 5 = \ln(2x + 3)
\Rightarrow x = \frac{\ln(2x + 3)}{\ln 5}
\]

Таким образом:
\[
\varphi(x) = \frac{\ln(2x + 3)}{\ln 5}
\]

Эта φ(x) имеет производную меньше 1 в нужной области и не вызывает переполнения.

---

## 📋 Результаты

| Метод                  | Уравнение                         | Приближённый корень | Кол-во итераций |
|------------------------|-----------------------------------|----------------------|------------------|
| Бисекция               | \( x^4 - 18x^2 + 6 = 0 \)         | ≈ 0.58301            | 9                |
| Простой итерации       | \( 5^x - 2x - 3 = 0 \)            | ≈ 1.00001            | 7                |
| Ньютона                | \( 5^x - 2x - 3 = 0 \)            | ≈ 1.00000            | 5                |

---

## 🧠 Выводы и анализ

- Метод бисекций всегда сходится, но **очень медленно**.
- Метод простой итерации с плохо выбранной φ(x) может **не сходиться вовсе**. Подбор φ(x) — ключевой этап!
- Метод Ньютона дал **наилучшую скорость сходимости** — всего 5 итерации против 7 у простой итерации.
- Аналитическое преобразование уравнения помогает избежать ошибок переполнения и ускоряет сходимость.